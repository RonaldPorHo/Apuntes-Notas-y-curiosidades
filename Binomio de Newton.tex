\documentclass[amssymb,12pt]{article}
\usepackage[utf8]{inputenc}
\usepackage[pangram]{blindtext}
\usepackage{geometry}
\usepackage{dsfont}
\geometry{letterpaper}
\usepackage{amsmath}
\usepackage{amssymb}
\usepackage{amsfonts}
\usepackage{xfrac}
\usepackage{vmargin}
\setpapersize{A4}
\setmargins{2.5cm}       % margen izquierdo
{1cm}                        % margen superior
{16.5cm}                      % anchura del texto
{23.42cm}                    % altura del texto
{10pt}                           % altura de los encabezados
{1cm}                           % espacio entre el texto y los encabezados
{0pt}                             % altura del pie de página
{2cm}                           % espacio entre el texto y el pie de página
\usepackage{amstext}
\usepackage{tcolorbox}
\usepackage{amsthm}
\usepackage{mathrsfs}
\usepackage{parskip}
\usepackage{xcolor}
\usepackage{fancybox}
\usepackage{xspace}
\usepackage{array,booktabs}
\usepackage{proof}
\usepackage{fancyhdr}
 \usepackage{paralist}
 \usepackage{hyperref}
\usepackage{marginnote}
\newcommand{\sen}{\operatorname{sen}}
\newcommand{\gr}{^{\circ}}
\usepackage{tkz-fct}
\usepackage{subcaption}
\usepackage{graphicx}
\usepackage{pstricks-add}
\usepackage{blindtext}
\usepackage{multido}
\usepackage{pstricks,pst-xkey}
\definecolor{ShadowColor}{RGB}{20,200,10}
\makeatletter
\define@key[psset]{}{rotAngle}{\pst@getangle{#1}\psk@rotAngle }
\psset{rotAngle=0}
%
\def\quadrat{\pst@object{quadrat}}
\def\quadrat@i#1{%
   \pssetlength\pst@dima{#1}	  
   \begin@ClosedObj              
   \addto@pscode{                 
     \psk@rotAngle rotate 	  
     0 0 moveto                   
     /s \pst@number\pst@dima def  
     s 0 rlineto                  
     0 s rlineto                 
     s neg 0 rlineto   		 
     0 s neg rlineto             
   }% 				   
   \end@ClosedObj% 		  
}
\makeatletter
\newcommand\Cshadowbox{\VerbBox\@Cshadowbox}
\def\@Cshadowbox#1{%
  \setbox\@fancybox\hbox{\fbox{#1}}%
  \leavevmode\vbox{%
    \offinterlineskip
    \dimen@=\shadowsize
    \advance\dimen@ .5\fboxrule
    \hbox{\copy\@fancybox\kern.5\fboxrule\lower\shadowsize\hbox{%
      \color{ShadowColor}\vrule \@height\ht\@fancybox \@depth\dp\@fancybox \@width\dimen@}}%
    \vskip\dimexpr-\dimen@+0.5\fboxrule\relax
    \moveright\shadowsize\vbox{%
      \color{ShadowColor}\hrule \@width\wd\@fancybox \@height\dimen@}}}
\makeatother
\begin{document}
\pagestyle{empty}
\begin{center}
\tcbset{colframe=green!50!black,colback=green!25,colupper=green!30!black,fonttitle=\bfseries,center title, nobeforeafter, tcbox raise base}
\begin{tcolorbox}
\begin{center}
 \textbf{\textcolor{red}{Demostración del Teorema del Binomio con Ecuaciones Diferenciales:}}
\end{center}
 \emph{Teorema:} Para cualquier numero real $r$ y $x,y$ reales se tiene:
 \[(x+y)^r=\sum_{m=0}^{\infty}\binom{r}{m}x^my^{r-m}\]
\end{tcolorbox}
\end{center}
\begin{tcolorbox}[title=\emph{Demostración:},
 colback=blue!5!white,colframe=blue!75!white]
La serie anterior converge si $|x|<|y|$. Entonces hacemos $t=\sfrac{x}{y}$. Entonces:
\[(1+t)^r=\sum_{m=0}^{\infty}\binom{r}{m}t^m \]
Ahora consideremos:
\[f_1(t)=(1+t)^r\quad\mathrm{y}\quad f_2(t)=\sum_{m=0}^{\infty}\binom{r}{m}t^m \]
Derivamos $f_1$ respecto a $t$ entonces:
\[f_1'(t)=r(1+t)^{r-1}\Rightarrow (1+t)f_1'(t)=rf_1(t) \]
Por tanto $f_1(t)$ es una solución de:
\[(1+t)y'=ry,\quad y(0)=1 \]
Ahora calculemos $(1+t)f_2'(t)$:
\[f_2'(t)=\sum_{m=1}^{\infty}\binom{r}{m}mt^{m-1}=r\sum_{m=1}^{\infty}\binom{r-1}{m-1}t^{m-1} \]
\[\begin{aligned}
\Rightarrow r(1+t) \sum_{m=1}^{\infty}\left(\begin{array}{c}
r-1 \\
m-1
\end{array}\right) t^{m-1} &=r \sum_{m=1}^{\infty}\left(\begin{array}{c}
r-1 \\
m-1
\end{array}\right) t^{m}+r \sum_{m=1}^{\infty}\left(\begin{array}{c}
r-1 \\
m-1
\end{array}\right) t^{m-1} \\
&=r \sum_{m=1}^{\infty}\left(\begin{array}{c}
r-1 \\
m-1
\end{array}\right) t^{m}+r \sum_{m=0}^{\infty}\left(\begin{array}{c}
r-1 \\
m
\end{array}\right) t^{m} \\
&=r \sum_{m=1}^{\infty}\left[\left(\begin{array}{c}
r-1 \\
m-1
\end{array}\right)+\left(\begin{array}{c}
r-1 \\
m
\end{array}\right)\right] t^{m}+r \\
&=r \sum_{m=1}^{\infty}\left(\begin{array}{c}
r \\
m
\end{array}\right) t^{m}+r=r \sum_{m=0}^{\infty}\left(\begin{array}{c}
r \\
m
\end{array}\right) t^{m} \\
&=r f_{2}(t)
\end{aligned} \]
Se ve que $f_2(0)=1$. Por tanto, empleando la unicidad de soluciones para ecuaciones diferenciales con valor inicial se obtiene que $f_1(t)=f_2(t)$

\end{tcolorbox}
%\vfill
%%\tiny{*Tanto los problemas como las soluciones no son %necesariamente adjudicadas al creador del documento, este fue %creado con el propósito de informar y entretener}








\end{document}