\documentclass[12pt]{article}
\usepackage[utf8]{inputenc}
\usepackage[T1]{fontenc}
\usepackage{geometry}
\geometry{letterpaper}
\usepackage{amsmath,amssymb,amsfonts}
\usepackage{graphicx}
\usepackage{parskip}
\usepackage{xcolor}
\usepackage{fancyhdr}
\usepackage{hyperref}
\usepackage{setspace}

\makeatletter
\newcommand\Cshadowbox{\VerbBox\@Cshadowbox}
\def\@Cshadowbox#1{%
  \setbox\@fancybox\hbox{\fbox{#1}}%
  \leavevmode\vbox{%
    \offinterlineskip
    \dimen@=\shadowsize
    \advance\dimen@ .5\fboxrule
    \hbox{\copy\@fancybox\kern.5\fboxrule\lower\shadowsize\hbox{%
      \color{ShadowColor}\vrule \@height\ht\@fancybox \@depth\dp\@fancybox \@width\dimen@}}%
    \vskip\dimexpr-\dimen@+0.5\fboxrule\relax
    \moveright\shadowsize\vbox{%
      \color{ShadowColor}\hrule \@width\wd\@fancybox \@height\dimen@}}}
\makeatother
\setlength{\topmargin}{-30mm}
\setlength{\bottommargin}{\-}


\pagestyle{fancy}
\begin{document}

\runningheadrule
\runningfootrule
\onehalfspacing
\newcommand{\tf}[1][{}]{%
\fillin[#1][0.25in]%
}
\begin{center}
    \begin{minipage}{1.7cm}
		\begin{center}
			\hspace*{-2cm} \includegraphics[scale=0.05]{uni-logo-color}
		\end{center}
	\end{minipage}
	\begin{minipage}{11.4cm}
		\begin{flushleft}
				\hspace*{-1.0cm} \small \text{Universidad Nacional de Ingenieria}\\ \hspace*{-0.85cm}\text{Facultad de Ingenieria Mecanica}\\
\hspace*{-0.85cm}\text{Departamento de Ciencias Basicas y Humanidades}\\ \hspace*{-0.85cm}\text{Ecuaciones Diferenciales MB 155} \hfill \text{Ciclo:2018-II}

		\end{flushleft}
	\end{minipage}

\end{center}

\pagestyle{empty}

\begin{center}
\textbf{\large Examen Final}
\end{center}

\begin{small}
Profesores: Carlos ROJAS, Manuel KUROKAWA, Sergio QUISPE, Victor HUANCA \quad
Seccion: A,B,C,D,E \hfill Duracion: 110 minutos \\
Fecha: Martes, 11 de diciembre del 2018 \hfill Hora: 08 a 10 horas
\end{small}

\makebox[1pt][l]{\rule[3pt]{15cm}{1pt}}\rule{15cm}{1pt}\\
\underline{Indicaciones:}\hfill Puntaje: Cada pregunta vale 5 puntos\\
Sin elementos de consulta \hfill Sin calculadoras, sin celulares\\

\textbf{Problema 1:}\\
Hallar:
\[
\text{a)}\ \mathcal{L}\big[|\sin t|\big]\hspace{80pt}
\text{b)}\ \mathcal{L}^{-1}\!\left[\frac{s^{3}+3s^{2}-s-3}{(s^{2}+2s+5)^{2}}\right]
\]

\textbf{Problema 2:}\\
Resolver la ecuación diferencial:
\[
y''+4y=f(x),\qquad y(0)=1,\qquad y'(0)=1
\]
donde
\[
f(x)=
\begin{cases}
0,& 0\le x<\pi,\\
1-\sin(3x),& x\ge \pi.
\end{cases}
\]

\textbf{Problema 3:}\\
Dada la función $f(t)=t^2$ para $0<t<2\pi$ y $f(t+2\pi)=f(t)$:
\begin{enumerate}
\renewcommand{\theenumi}{\alph{enumi}}
\renewcommand{\labelenumi}{\theenumi)}
\item Encuentre la serie de Fourier de $f(t)$.
\item Con lo anterior, evalúe $\displaystyle\sum_{n=1}^{\infty}\frac{(-1)^{n}+1}{n^2}$.
\end{enumerate}

\textbf{Problema 4:}\\
Los extremos de una barra de cobre de $2$ m de longitud (coeficiente de difusividad $c=1.14$) se mantienen a $0^{\circ}C$. Si la temperatura inicial es
\[
f(x)=\cos^{2}(\pi x),\qquad 0\le x\le 2,
\]
donde $x$ es la distancia desde un extremo, encuentre la expresión de la temperatura de la barra después de un tiempo $t$.

\end{document}
