\documentclass[12pt]{article}
\usepackage{geometry}
\geometry{letterpaper}
\usepackage[utf8]{inputenc}
\usepackage[spanish]{babel}
\usepackage[T1]{fontenc}
\usepackage{amsmath}
\usepackage{amssymb}
\usepackage{amsfonts}
\usepackage{amstext}
\usepackage{amsthm}
\usepackage{mathrsfs}
\usepackage{amsmath}
\usepackage{graphicx}
\usepackage{parskip}
\usepackage{xcolor}
\usepackage{fancybox}
\definecolor{ShadowColor}{RGB}{200,10,10}
\usepackage{fancybox}	
\usepackage{xspace}
\usepackage{array,booktabs}
\usepackage{proof}
\usepackage{fancyhdr}
 \usepackage{paralist}
 \usepackage{hyperref}
\usepackage{marginnote}
\newcommand{\sen}{\operatorname{sen}}
\makeatletter
\newcommand\Cshadowbox{\VerbBox\@Cshadowbox}
\def\@Cshadowbox#1{%
  \setbox\@fancybox\hbox{\fbox{#1}}%
  \leavevmode\vbox{%
    \offinterlineskip
    \dimen@=\shadowsize
    \advance\dimen@ .5\fboxrule
    \hbox{\copy\@fancybox\kern.5\fboxrule\lower\shadowsize\hbox{%
      \color{ShadowColor}\vrule \@height\ht\@fancybox \@depth\dp\@fancybox \@width\dimen@}}%
    \vskip\dimexpr-\dimen@+0.5\fboxrule\relax
    \moveright\shadowsize\vbox{%
      \color{ShadowColor}\hrule \@width\wd\@fancybox \@height\dimen@}}}
\makeatother



\begin{document}
\pagestyle{empty}
\begin{center}
\textcolor{red}{Un ejemplo clásico para la integral de Riemann :)}
\end{center}
\begin{center}
\Cshadowbox{Calcular: $\displaystyle\int_{0}^{\pi}\dfrac{\sin^2 rx}{\sin^2 \frac{1}{2}x}dx, \quad r\geq0$}
\end{center}
\begin{center}
\textbf{Solución:}
\end{center}
Resolveremos la integral mediante sumas de Riemann, recordemos que:
\begin{equation}
\lim_{n\rightarrow \infty}\dfrac{\pi}{n} \sum_{k=1}^{n} \dfrac{\sin^2 (kr\pi/n)}{\sin^2(k\pi/2n)} 
\end{equation}
Ahora note que para $m\geq2$ :
\begin{equation*}
1-\cos mx=1-\cos x+\sum_{p=1}^{m-1} [\cos px-\cos(p+1)x]
\end{equation*}
\begin{equation*}
\cos px-\cos(p+1)x=1-\cos x-\sum_{q=1}^{p} [\cos(q-1)x-2\cos qx+\cos(q+1)x]
\end{equation*}
\begin{equation*}
\cos(q-1)x+\cos(q+1)x=2\cos qx \cos x
\end{equation*}
Entonces:
\begin{equation*}
1-\cos mx=(1-\cos x)\left[m+2\sum_{p=1}^{m-1}\sum_{q=1}^{p}\cos qx \right]=(1-\cos x)\left[m+2\sum_{j=1}^{m-1}(m-j)\cos jx \right]
\end{equation*}
\begin{equation*}
\Rightarrow \dfrac{\sin^2 \frac{1}{2}mx}{\sin^2 \frac{1}{2}x}=\dfrac{1-\cos mx}{1-\cos x}=m+2\sum_{j=1}^{m-1}(m-j)\cos jx, \qquad 0<\dfrac{1}{2}x<\pi
\end{equation*}
\qquad Con $m=2r>0$ y $x=k\pi/n$ se obtiene algo similar a lo buscado en (1):
\begin{equation}
\sum_{k=1}^{n-1}\left[2r+2\sum_{j=1}^{2r-1} (2r-j)\cos (\dfrac{jk\pi}{n})\right]=2r(n-1)+2\sum_{j=1}^{2r-1}(2r-j)\sum_{k=1}^{n-1}\cos \dfrac{jk\pi}{n}
\end{equation}
\qquad Usamos la fórmula:
\begin{equation*}
\cos x+\cos 2x+'cdots+\cos mx=\dfrac{\sin(m+\frac{1}{2})x}{2\sin (\frac{1}{2}x)}-\dfrac{1}{2}
\end{equation*}
\end{document}