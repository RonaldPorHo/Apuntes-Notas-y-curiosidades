\documentclass[12pt]{article}
\usepackage[utf8]{inputenc}
\usepackage{geometry}
\usepackage{dsfont}
\geometry{letterpaper}
\usepackage{amsmath}
\usepackage{amssymb}
\usepackage{amsfonts}
\usepackage{amstext}
\usepackage{amsthm}
\usepackage{mathrsfs}
\usepackage{parskip}
\usepackage{xcolor}
\usepackage{fancybox}
\usepackage{xspace}
\usepackage{array,booktabs}
\usepackage{proof}
\usepackage{fancyhdr}
 \usepackage{paralist}
 \usepackage{hyperref}
\usepackage{marginnote}
\newcommand{\sen}{\operatorname{sen}}
\newcommand{\gr}{^{\circ}}
\usepackage{tkz-fct}
\usepackage{subcaption}
\usepackage{graphicx}
\usepackage{pstricks-add}
\usepackage{blindtext}
\usepackage{multido}
\usepackage{pstricks,pst-xkey}
\definecolor{ShadowColor}{RGB}{20,200,10}
\makeatletter
\define@key[psset]{}{rotAngle}{\pst@getangle{#1}\psk@rotAngle }
\psset{rotAngle=0}
%
\def\quadrat{\pst@object{quadrat}}
\def\quadrat@i#1{%
   \pssetlength\pst@dima{#1}	  
   \begin@ClosedObj              
   \addto@pscode{                 
     \psk@rotAngle rotate 	  
     0 0 moveto                   
     /s \pst@number\pst@dima def  
     s 0 rlineto                  
     0 s rlineto                 
     s neg 0 rlineto   		 
     0 s neg rlineto             
   }% 				   
   \end@ClosedObj% 		  
}
\makeatletter
\newcommand\Cshadowbox{\VerbBox\@Cshadowbox}
\def\@Cshadowbox#1{%
  \setbox\@fancybox\hbox{\fbox{#1}}%
  \leavevmode\vbox{%
    \offinterlineskip
    \dimen@=\shadowsize
    \advance\dimen@ .5\fboxrule
    \hbox{\copy\@fancybox\kern.5\fboxrule\lower\shadowsize\hbox{%
      \color{ShadowColor}\vrule \@height\ht\@fancybox \@depth\dp\@fancybox \@width\dimen@}}%
    \vskip\dimexpr-\dimen@+0.5\fboxrule\relax
    \moveright\shadowsize\vbox{%
      \color{ShadowColor}\hrule \@width\wd\@fancybox \@height\dimen@}}}
\makeatother
\begin{document}
\emph{La soluciones no son adjudicadas a mi persona, estas fueron recopiladas por lo que provienen de diversas fuentes}
\begin{center}
    \Large{Soluciones}
\end{center}
\begin{enumerate}
    \item[69.] \small{Encuentre la curva plana que pasa por el punto $M_0$ (2,4) que cumple lo siguiente: dibujamos dos líneas rectas a través de cualquier punto de la curva, paralela a los ejes de coordenadas. Entonces, el área de una de dos superficies planas, determinadas por este rectángulo y la curva es dos veces más grande que la otra}
    \[\text{Área} OAMC=2 \text{Area} CBM\]
    \[OAMC=\int_{0}^{x}ydx \]
    \[ CBM=xy-\int_{0}^{x}ydy\]
    \[\Rightarrow \int_{0}^{x}ydx=2(xy-\int_{0}^{x}ydy\]
    \[3 \int_{0}^{x} y d x=2 x y \Rightarrow 3 y=3 y+2 x y^{\prime}
\Rightarrow 2 x y^{\prime}=y \]
    \[y^2=Cx,y(2)=4 \therefore C=8,y^2=8x \]
    \item [38.] Si $\mathcal{L}[F(u)]=f(s)$. Calcular: \[\mathcal{L}\left[\int_{0}^{1} J_0(2\sqrt{u(t-u)})F(u)du\right] \]
    \[J_0(t)=\sum_{n=0}^{\infty} \dfrac{(1)^k}{(k!)^2}\left(\dfrac{t}{2}\right)^{2k} \Rightarrow J_{0}(2 \sqrt{u(t-u)})=\sum_{n=0}^{\infty} \frac{(-1)^{k}[u(t-u)]^{k}}{(k !)^{2}}=\sum_{k=0}^{\infty} \frac{(-1)^{k} u^{k}(t-u)^{k}}{(k!)^{2}} \]
    \[\Rightarrow J_{0}(2 \sqrt{u(t-u)}) F(u)=\sum_{k=0}^{\infty} \frac{(-1)^{k}}{(k t)^{2}} u^{k}(t-u)^{k} F(u) \]
    \[\Rightarrow \int_{0}^{t} J_{0}(2 \sqrt{u(t-u)}) F(u) d u=\sum_{k=0}^{\infty} \frac{(-1)^{k}}{(k!)^{2}} \int_{0}^{t} u^{k}(t-u)^{k} F(u) d u \]
    \[\Rightarrow \mathcal{L}\left[\int_{0}^{t} J_{0}(2 \sqrt{u(t-u)}) F(u) du \right]=\sum_{k=0}^{\infty} \frac{(-1)^{k}}{(k !)^{2}} \mathcal{L}\left[\int_{0}^{t} u^{k}(t-u)^{k} F(u) d u\right] \]
    Por convolución:
    \[=\sum_{k=0}^{\infty} \frac{\left(-1)^{k}\right.}{(k-1)^{2}} \mathcal{L}\left[t^{k} \cdot t^{k} F(t)\right]=\sum_{k=0}^{\infty} \frac{(-1)^{k}}{(k !)^{2}}\left[\frac{k !}{s^{k+1}}(-1)^{k} \frac{d^{k}}{d s^{k}}(f(s))\right] \]
    \[=\sum_{k=0}^{\infty} \frac{(-1)^{2 k}}{(k-1)^{2}}\left(\frac{k !}{s^{k+1}}\right) f^{(k)}(s)\] 
    \begin{center}
        \Cshadowbox{$=\displaystyle\sum_{k=0}^{\infty} \frac{k !}{(k !)^{2}} \frac{f^{(k)}(s)}{s^{k+1}}$}
    \end{center}
    \item[37.]
    \[D=\displaystyle\dfrac{\displaystyle\int_{0}^{1} \dfrac{\ln (x) \ln (1-x)}{x} d x}{\displaystyle\int_{-\infty}^{\infty} \arctan \left(e^{x}\right) \arctan \left(e^{-x}\right) d x} \]
    En la primera integral:
    \[\int_{0}^{1} \frac{\ln x \ln (1-x)}{x} d x=-\sum_{n \geq 1} \frac{1}{n} \int_{0}^{1} \ln x x^{n-1} d x=\sum_{n \geq 1} \frac{1}{n^{3}}=\zeta(3)\]
    Usando integración por partes se puede probar:
    \[\int_{0}^{\pi} u(\pi-u) e^{-(2 k+1) u i} d u=-\frac{4 i}{(2 k+1)^{3}} \qquad k \in \mathbb{N} \cup\{0\} \]
    \[\Rightarrow \int_{0}^{\pi} \frac{u(\pi-u)}{\sin u}\left(1-e^{-2 K u i}\right) d u=2 i \sum_{k=0}^{K-1} \int_{0}^{\pi} u(\pi-u) e^{-(2 k+1) u i} d u=8 \sum_{k=0}^{K-1} \frac{1}{(2 k+1)^{3}} \]
    Usando $K\to\infty$, se deduce que:
    \[\int_{0}^{\pi} \frac{u(\pi-u)}{\sin u} d u=8 \sum_{k \geq 0} \frac{1}{(2 k+1)^{3}}=8\left(1-\frac{1}{8}\right) \zeta(3)=7 \zeta(3)\]
    Ahora en la integral del denominador hacemos $y=e^x$:
    \[\int_{0}^{\infty} \frac{\tan ^{-1}(y) \tan ^{-1}\left(y^{-1}\right)}{y} d y \]
    Luego $y=\tan t$:
    \[\int_{0}^{\frac{1}{2} \pi} \frac{t\left(\frac{1}{2} \pi-t\right)}{\sin t \cos t} d t \]
    Finalmente $u=2t$:
    \[=\frac{1}{4} \int_{0}^{\pi} \frac{u(\pi-u)}{\sin u} d u=\frac{7}{4} \zeta(3)\]
    \begin{center}
        \Cshadowbox{$\therefore D=\dfrac{4}{7} $}
    \end{center}
    \item [39] Encuentre la familia de integrales de superficie de la siguiente ecuación diferencial parcial no lineal de orden uno
    \[p^2+q^2=f(\sqrt{x^2+y^2})\]
    Usamos polares entonces:
    \[x=r\cos\theta \quad y=r\sen \theta \]
    \[
\dfrac{\partial u}{\partial r}=\dfrac{\partial u}{\partial x} \cos \theta+\dfrac{\partial u}{\partial y} \sen \theta \] \[
\dfrac{\partial u}{\partial \theta}=-\dfrac{\partial u}{\partial x} r \sen \theta+\dfrac{\partial u}{\partial y} r \cos \theta\]
Usando esto:
\[
\dfrac{\partial u}{\partial x}=\cos \theta \dfrac{\partial u}{\partial r}-\dfrac{\sin \theta}{r} \dfrac{\partial u}{\partial \theta} \] \[
\dfrac{\partial u}{\partial y}=\sin \theta \dfrac{\partial u}{\partial r}+\dfrac{\cos \theta}{r} \dfrac{\partial u}{\partial \theta}
 \]
 Entonces la ecuación inicial se puede reescribir:
 \[\left(\frac{\partial u}{\partial r}\right)^{2}+\frac{1}{r^{2}}\left(\frac{\partial u}{\partial q}\right)^{2}=f(r) \]
 Entonces el sistema característico:
 \[
\frac{d r}{2 p r^{2}}=\frac{d \theta}{2 q}=\frac{d u}{2 p^{2} r^{2}+2 q^{2}}= \\
=\frac{-d p}{2 r p^{2}-2 r f-r^{2} f^{\prime}}=\frac{-d q}{0}
\]
Donde $p=\dfrac{\partial u}{\partial r},q=\dfrac{\partial u}{\partial \theta}\Rightarrow q=a$. De la ecuación:
\[p=\pm \sqrt{f(r)-\frac{a^{2}}{r^{2}}} \]
Teniendo la ecuación Pfaff es:
\[d u=\pm \sqrt{f(r)-\frac{a^{2}}{r^{2}}} d r+a d \theta \]
Y la integral completa sería:
\[u=\pm \int_{r_{0}}^{\sqrt{x^{2}+y^{2}}} \sqrt{f(r)-\frac{a^{2}}{r^{2}}}+a \arctan \frac{y}{x}+b, \qquad a, b \in \mathbb{R} \]
\end{enumerate}










\end{document}