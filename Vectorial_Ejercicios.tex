\documentclass[12pt]{article}
\usepackage[utf8]{inputenc}
\usepackage{geometry}
\geometry{letterpaper}
\usepackage{amsmath, amssymb, amsfonts, amsthm}
\usepackage{parskip}
\usepackage{xcolor}  
\usepackage{array,booktabs}
\usepackage{enumitem} 
\usepackage{graphicx}
\usepackage{subcaption}
\usepackage{hyperref}
\newcommand{\sen}{\operatorname{sen}}
\newcommand{\gr}{^{\circ}}
\begin{document}
\begin{enumerate}

    \item Sea $f:U\to\mathbb{R}$ diferenciable en $U\subset\mathbb{R}^m$ y abierta. Suponga $df(a)\neq0$ para cierto $a\in U$ y considere el vector unitario $u\in\mathbb{R}^m$ tal que $df(a)\cdot u=\max\{df(a)\cdot h;|h|=1\}$. Sea $v\in\mathbb{R}^m$ tal que $df(a)\cdot v=0$ muestre que $v$ es perpendicular a $u$.\\

    \emph{Solución:}\\
    Se tiene: $df(a)\cdot u\geq df(a)\cdot h,\ \forall h/|h|=1$. En particular:
    \[
      h=\dfrac{\nabla f(a)}{|\nabla f(a)|}
    \]
    \[
      \Rightarrow df(a)\cdot u \geq df(a)\cdot \frac{\nabla f(a)}{|\nabla f(a)|}
      =\left\langle\nabla f(a), \frac{\nabla f(a)}{|\nabla f(a)|}\right\rangle
      =|\nabla f(a)|
    \]
    \[
      df(a)\cdot u=\langle\nabla f(a), u\rangle \leq|\nabla f(a)|\cdot|u|
      =|\nabla f(a)|
    \]
    \[
      \therefore df(a)\cdot u=|\nabla f(a)|
    \]
    \[
      u=\alpha \nabla f(a)\Rightarrow \alpha=\pm1
    \]
    \[
      df(a)\cdot v=0 \ \Rightarrow\ \langle\nabla f(a), v\rangle=0 \ \Rightarrow v \perp \nabla f(a)
    \]
    \[
      \nabla f(a)//u \ \Rightarrow u\perp v
    \]

    \item Muestre que $\dfrac{d \mathbf{r}}{d s} \cdot \dfrac{d^{2} \mathbf{r}}{d s^{2}} \times \dfrac{d^{3} \mathbf{r}}{d s^{3}}=\dfrac{\tau}{\rho^{2}}$:  

    \[
    \frac{d \mathbf{r}}{d s}=\mathbf{T}, \quad 
    \frac{d^{2} \mathbf{r}}{d s^{2}}=\kappa \mathbf{N}, \quad 
    \frac{d^{3} \mathbf{r}}{d s^{3}}=\kappa \tau \mathbf{B}-\kappa^{2} \mathbf{T}+\frac{d \kappa}{d s} \mathbf{N}
    \]

    \[
    =\mathbf{T}\cdot\big(\kappa^{2}\tau \mathbf{T}+\kappa^{3}\mathbf{B}\big)
    =\kappa^{2}\tau=\frac{\tau}{\rho^{2}}, \quad \rho=\frac{1}{\kappa}
    \]

    \item Dada la ecuación $x=t, \ y=t^2, \ z=\tfrac{2}{3}t^3$, calcule la curvatura y la torsión.

    \begin{enumerate}[label=(\alph*)]
        \item 
        \[
          \mathbf{r}=t\mathbf{i}+t^2\mathbf{j}+\frac{2}{3}t^3\mathbf{k}, \quad 
          \frac{d \mathbf{r}}{dt}=\mathbf{i}+2t\mathbf{j}+2t^2\mathbf{k}
        \]
        \[
          \frac{ds}{dt}=\sqrt{1+(2t)^2+(2t^2)^2}=1+2t^2
        \]
        \[
          \mathbf{T}=\frac{\mathbf{i}+2t\mathbf{j}+2t^2\mathbf{k}}{1+2t^2}
        \]
        \[
          \frac{d\mathbf{T}}{ds}=\frac{-4t\mathbf{i}+(2-4t^2)\mathbf{j}+4t\mathbf{k}}{(1+2t^2)^3}
        \]
        \[
          \kappa=\left|\frac{d\mathbf{T}}{ds}\right|=\frac{2}{(1+2t^2)^2}
        \]

        \item 
        \[
          \mathbf{B}=\mathbf{T}\times \mathbf{N}
          =\frac{2t^2 \mathbf{i}-2t \mathbf{j}+\mathbf{k}}{1+2t^2}
        \]
        \[
          \frac{d\mathbf{B}}{ds}=\frac{4t\mathbf{i}+(4t^2-2)\mathbf{j}-4t\mathbf{k}}{(1+2t^2)^3}
        \]
        \[
          \tau=\frac{2}{(1+2t^2)^2}
        \]
    \end{enumerate}
\end{enumerate}

\end{document}
